\chapter{Discussion}
\label{chapterlabel7}

In the results, the simulations of different model assumptions has been shown. The most significant differences can been seen on the model using unsmoothened aorta followed by the two geometries that were expanded and narrowed. While the unsmoothened geometry was still classified as "good", the differences can be seen on both the outlet condition and the haemodynamic indices. where the TAWSS values show to be much higher then in the other simulation. The choice of the image processing algorithm is often dependent on the decision taken by the modeler. Additionally due to the coupling of the 0D model to the 3D model, due to the varying area of the lumen, the Windkessel parameter tuned to the geometry would yield a different result in the 3D CFD simulation.\par

The modelling choices concerning images has shown to be a significant factor that influence the outcome of the CFD. In numerous CFD challenges, where the models were reconstructed from the images by a number of groupd, the image processing pipeline differed from group to group resulting in differences of TAWSS up to 40\% \cite{Steinman2018Editorial:Utility, Huberts2018WhatPaper}.\par

A comparison of the different blood modelling assumptions has shown very small differences between the models. The Newtonian fluid approximation is often widely accepted as the variation between the viscosity model are very minor, especially when compared with the effect of the image-based variations \cite{Steinman2019HowVariability, Lee2007OnBifurcation}. However in using a non-Newtonian blood can give important details on flow especially in the cases of smaller arteries and aneurysm studies \cite{Johnston2006Non-NewtonianSimulations, Steinman2012AssumptionsHemodynamics}. While there has been some studies of viscosity's influence on the blood flow in larger arteries, there is no objective reference that could quantify the variability therefore, the results were often interpreted subjectively \cite{Steinman2012AssumptionsHemodynamics}.\par

The different meshes in our simulations have produced very similar results. The study case in this project is relatively simple, however a more complex geometries would often require an appropriately sized mesh and a good quality. The meshes then vary from case to case therefore proper guidelines are difficult to provide. Furthermore, due to a number of different solvers, a numerical uncertainty factor is involved as well which is often hard to quantify as most of the CFD solver settings are hidden in the commercial software. \par