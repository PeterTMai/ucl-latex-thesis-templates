\chapter{Future Works}
\label{chapterlabel9}

While this project has to explored the different uncertainties that are involved in a CFD modelling of blood flow models, the study has been simplified as a proof of concept of the framework to identify the sources of biggest uncertainties in cardiovascular modelling. In this chapter the further work that will need to be done in order to create proper guidelines to use to the framework.

\section{Additional assumptions to consider}
In this project the image processing, segmentation, mesh size and the blood viscosity has been considered to quantify their effects on the simulation. However, there is a number of the other factors that can give rise to variability in the simulations.

\subsection{Rigid wall vs. Compliant wall}
While most of the CFD studies use the assumption that the walls are rigid wall, the walls of arteries have elastic properties. A promising method modelling the wall motion that is less computationally expensive then FSI would be the moving boundary method that models the wall motion based on the images of 2D cine-MRI \cite{Bonfanti2018AInteraction}. 

\subsection{Boundary conditions}
In the Chapter \ref{chapterlabel2}, different inflow assumptions have been discussed. However, in this project only the uniform profile has been used. A number of different inlet profile can be used for subsequent analysis. \par

In addition to the Windkessel model at the outlet, a constant pressure models can be compared as well to quantify the flow rates and pressure at the outlets as well.

\section{Variability resulting from interactions}
In this project, the assumptions were evaluated individually. However it is important to note that additional variability could be the result of the different assumption interaction as well. Thus, a much more comprehensive study of all the different permutations need to be done.

\section{Haemodynamic parameters}
In this study the studied haemodynamic parameter was TAWSS. However, the 