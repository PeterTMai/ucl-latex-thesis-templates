\chapter{Results}
\label{chapterlabel6}

One case from the permutations of the simulations is being used as the control model. The case was taken based on most common assumptions in the current research, which is the case of a normal aorta, smoothened wall and with Newtonian fluid. The mesh selected was the mesh 3(~ 600,000 elements).

\section{Outlet flow and pressure comparison}
A comparison of the outlet flow conditions were compared. As there was no patient data on the flow conditions apart from the flow rate, the percentage difference between the models has been calculated as:
\begin{align}
    \% diff = \frac{\bar{Q}_{mod1}-\bar{Q}_{mod2}}{\frac{\bar{Q}_{mod1}-\bar{Q}_{mod2}}{2}} \times 100 \%
\end{align}

A comparison of the different models show  that while the differences in pressures are very small, less then 1\%, the outlet flow can be result in very different results. \par

The most noticeable difference can be observed between the different segmentations of the patient aorta, where the mean difference of the flow at the outlet was 43.75\% followed by the image processing influence, where the difference was 34.05\%. The smallest difference was shown to be the influence of the viscosity model, as the difference across the outlets was smaller then 2\%. The influence of the mesh also showed to have very little effect on the outlet conditions where the flow rate differences were on average 2.64\%. \par

\section{Haemodynamic indices}

