\chapter{Discussion}
\label{chapterlabel7}


From the results, the areas that could be ranked as most uncertain would be the assumptions that are related to imaging. The most significant differences can been seen on the model using unsmooth geometry followed by the two geometries that were dilated. While the unsmooth geometry was still classified as "good", the differences can be seen on both the outlet condition and the haemodynamic indices, where the TAWSS values are much higher then in the other simulation. These differences can also be seen on the dilated models. Additionally, by comparing the models that used both assumptions, the results of TAWSS showed very significant variability as well. The modelling choices involving images has shown to be a significant factor that influence the outcome of the CFD as they can introduce significant error. In numerous CFD challenges, where the models were reconstructed from the images by a number of groups, the image processing pipeline differed from group to group resulting in differences of TAWSS up to 40\% \cite{Steinman2018Editorial:Utility, Huberts2018WhatPaper}.\par

A comparison of the different blood modelling assumptions shows very little variability, thus, the blood model could be considered the least uncertain from the considered assumptions. While non-Newtonian blood can give important details on flow especially in the cases of smaller arteries and aneurysm studies \cite{Johnston2006Non-NewtonianSimulations, Steinman2012AssumptionsHemodynamics}, Newtonian fluid approximation is often widely accepted in the simulation of the larger arteries. Additionally, the variation between the viscosity model are very minor, especially when compared with the effect of the image-based variations \cite{Steinman2019HowVariability, Lee2007OnBifurcation}. There has been some studies of viscosity's influence on the blood flow in larger arteries, however, there is no objective reference that could quantify the variability therefore, the results were often interpreted subjectively \cite{Steinman2019HowVariability}.\par

The mesh influence on the results is seen on the comparison of the flow curves, however the mean TAWSS can be seen to be similar. The choice of the mesh vary from case to case therefore proper guidelines are difficult to provide as a more complex would require a much more higher quality mesh with appropriately sized elements \cite{Hodis2012GridAneurysms}. Furthermore, due to a number of different solvers, a numerical uncertainty factor is involved as well which is often hard to quantify as most of the CFD solver settings are hidden in the commercial software. \par

