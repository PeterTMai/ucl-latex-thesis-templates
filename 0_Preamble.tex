% I may change the way this is done in a future version, 
%  but given that some people needed it, if you need a different degree title 
%  (e.g. Master of Science, Master in Science, Master of Arts, etc)
%  uncomment the following 3 lines and set as appropriate (this *has* to be before \maketitle)
% \makeatletter
% \renewcommand {\@degree@string} {Master of Things}
% \makeatother

\title{Structural Uncertainty in Cardiovascular CFD models}
\author{Thanh Trung Mai}
\department{Medical Physics and Biomedical Engineering}

\maketitle
\makedeclaration

\begin{abstract} % 300 word limit
My research is about stuff.

It begins with a study of some stuff, and then some other stuff and things.

There is a 300-word limit on your abstract.
\end{abstract}

\begin{impactstatement}

UCL theses now have to include an impact statement. \textit{(I think for REF reasons?)} The following text is the description from the guide linked from the formatting and submission website of what that involves. (Link to the guide: {\scriptsize \url{http://www.grad.ucl.ac.uk/essinfo/docs/Impact-Statement-Guidance-Notes-for-Research-Students-and-Supervisors.pdf}})

\begin{quote}
The statement should describe, in no more than 500 words, how the expertise, knowledge, analysis,
discovery or insight presented in your thesis could be put to a beneficial use. Consider benefits both
inside and outside academia and the ways in which these benefits could be brought about.

The benefits inside academia could be to the discipline and future scholarship, research methods or
methodology, the curriculum; they might be within your research area and potentially within other
research areas.

The benefits outside academia could occur to commercial activity, social enterprise, professional
practice, clinical use, public health, public policy design, public service delivery, laws, public
discourse, culture, the quality of the environment or quality of life.

The impact could occur locally, regionally, nationally or internationally, to individuals, communities or
organisations and could be immediate or occur incrementally, in the context of a broader field of
research, over many years, decades or longer.

Impact could be brought about through disseminating outputs (either in scholarly journals or
elsewhere such as specialist or mainstream media), education, public engagement, translational
research, commercial and social enterprise activity, engaging with public policy makers and public
service delivery practitioners, influencing ministers, collaborating with academics and non-academics
etc.

Further information including a searchable list of hundreds of examples of UCL impact outside of
academia please see \url{https://www.ucl.ac.uk/impact/}. For thousands more examples, please see
\url{http://results.ref.ac.uk/Results/SelectUoa}.
\end{quote}
\end{impactstatement}

\begin{acknowledgements}
Acknowledge all the things!
\end{acknowledgements}

\setcounter{tocdepth}{2} 
% Setting this higher means you get contents entries for
%  more minor section headers.

\tableofcontents
\listoffigures
\listoftables

