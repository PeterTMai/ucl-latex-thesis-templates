\chapter{Introduction}
\label{chapterlabel1}

Computational models are usually constructed by applying an established physical principle that relate time-dependent variables which describe the state of the system to measured model parameters. This makes the models suitable to study complex dynamic processes and systems and predict the future behaviour of the system.\par

Over the past decade, the use of computational modelling in the cardiovascular research has showed numerous different approaches that could be benefitial for clinical use. The use of computational models with patient-specific inputs calculates in both, space and time, variables such as flow, velocity gradients, pressures, wall shear stress and vessel motion displacement which can be helpful in decision-making processes. Additionally, the models can be parametrically modified to assist clinicians in investigating the outcome of various treatment methods, following a systematic approach.

Despite the numerous effort in making cardiovascular models viable for the clinical support, there are still numerous major challenges for these models to be used in the clinic. Often, patient-specific data in certain areas is difficult or impossible to obtain, thus simplifications of real-life processes would be necessary. Furthermore, due to this lack of data, often models are run in an "idealised" scenario which can introduce structural uncertainty into the results, which is the error stemming from inadequate assumptions. Lastly, there is a very little consensus in the cardiovascular modelling on the assessment of these uncertainties and variability. Nevertheless, these are critical to gain trust and acceptance in the clinic  \par

In this report, the background to cardiovascular modelling will be introduced and the process of verification, validation and uncertainty quantification of cardiovascular models will be explained. In the literature review, the different methods to assess uncertainties will be explained and current work in the cardiovascular modelling field  will be reviewed. Then a conceptual modelling framework to identify the structural uncertainty areas will be introduced.