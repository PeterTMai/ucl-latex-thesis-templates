\chapter{Introduction}
\label{chapterlabel1}

Computational models have been used in many research areas such as geology, physics or engineering to predict outcomes of complex problems. Computational models are usually constructed by applying an established physical principle that relate time-dependent variables which describe the state of the system to measured model parameters. This makes the models suitable to study the complex dynamic processes and system and predict the future behaviour of the system. In recent years, with the increase of computational power, fields such as biology or medicine, where the models can depend on a multitude of variables, have been using computational models as a framework to predict the behaviour of the systems and study the processes that are not fully understood.\par

In medical research, computational models have become a powerful tool to establish the fundamental understanding of the biology and anatomy in human body and also for medical device design. The consideration of using computational framework for diagnosis and interventional planning has been also gaining traction and numerous research areas have been emerged to create a tool that could assist the clinicians in their decision-making. One of these areas has been shown to be the the study of cardiovascular haemodynamics. \par

Over the past decade, the use of computational modelling in the cardiovascular research has showed numerous different approaches that could be benefitial for the clinical use. The use of computational models with patient-specific inputs can result in spatial and temporal data on parameters such as flow, velocity gradient, pressure, wall shear stress and vessel motion which could be used for decision making process. Additionally, the models can be parametrically modified to assist clinicians in investigating the outcome of various treatment methods.

Despite the numerous effort in making cardiovascular models viable for the clinical support, there are still numerous major challenges to these models viable in the use decision support. In fact, the choice of using computational models have been also dismissed due to these shortcomings. Often, patient-specific data in certain areas could be lacking thus simplifications of the real-life process would be necessary. Furthermore, due to the lack of data, often the models are run in an "idealized" scenario which can yield uncertain results. Lastly, there is a very little consensus in the cardiovascular modelling on the assessment of these uncertainties and variability.  \par